\documentclass[draft]{csuthesis} %draft means that it won't actually render any pictures (graphs) it will just include space for them.
\title[Improved Multifractal Calculations]{Improved Methods for Calculating the Multifractal Spectrum for Small Data Sets}
\author{Leif Anderson}
\departmentname{Department of Physics}
\gradterm{Spring 2014} %The semester that you are going to turn this in.
\advisor{Richard Eykholt}
\coadvisor{Test Test} %optional
\committee{Mingzong Wu \and Raymond Steve Robinson \and Martin Gelfand \and Patrick Shipman} %separate committee names with \and
%committee names should not have any honorifics (i.e. NO Dr., PhD, professor, etc.)  Just names.
\copyrighttext{Copyright by Leif Anderson 2014 \\ All Rights Reserved} %this is optional.

%you do not need these newcommand lines.
\newcommand{\deriv}[2]{\frac{\mathrm{d}#1}{\mathrm{d}#2}}
\newcommand{\dby}[1]{\frac{\mathrm{d}}{\mathrm{d}#1}}
\newcommand{\pd}[2]{\frac{\partial#1}{\partial#2}}
\newcommand{\pby}[1]{\frac{\partial}{\partial#1}}

\usepackage{lipsum}%this package provides nonsense text for testing document layouts.  Not needed for real thesis.

\begin{document}
\frontmatter
%turns the page numbering to roman, and does some other stuff.

% 
%Anomaly detection in video streams has been a task of interest for many years.
With the increasing number and variety of camera installations, unsupervised methods that learn typical activities have become popular for anomaly detection.
In this article, we consider recent methods based on temporal probabilistic models and improve them in multiple ways.
Our contributions are the following:
(i) we integrate the low level processing and the temporal activity modeling, showing how this feedback improves the overall quality of the captured information,
(ii) we show how the same approach can be taken to do hierarchical multi-camera processing,
(iii) we use spatial analysis of the anomalies both to perform local anomaly detection and to frame automatically the detected anomalies.
We illustrate the approach on both traffic data and videos coming from a metro station.


\begin{abstract}
This is the abstract right here. I wanted a short input, so I typed these couple sentences, then I included some latin filler: \lipsum[1-4]

End of Abstract.
\end{abstract}
%note that for amsbook (which is basically the class we are copying), the abstract must be declared before the title.  It prints as part of maketitle.

\begin{acknowledgements}
Thanks to Dan Brake, Fran Campana, and Natalie Anderson for their help defeating the rules of the graduate school; to Katherine Zaunbrecher for her proofreading and motivation help; and to many others.
\end{acknowledgements}%this is optional.  It must be declared before maketitle, a lot like abstract.  Consider having this in a separate file.

\maketitle
% maketitle is a huge command in a small package.  It will render the title page, copyright page, acknowledgements, abstract, and probably a bunch of other stuff that I'm forgetting about.  If you choose not to define copyright, acknowledgements, etc, it automatically doesn't render them.

% other optional frontmatter could go somewhere in here.  dedication, biography, etc.  Grad school lists approved frontmatter pages somewhere.
% most frontmatter than I consider silly is not yet automated.  Also most frontmatter doesn't have any explicit style requirements, so I feel justified in ignoring it.

\tableofcontents %don't move this around too much
\listoftables %optional, but this is the spot it should go
\listoffigures %optional, but this is the right location for it

\mainmatter %switches page numbering to normal, resets page counter, etc.
\chapter{Introduction}

Before beginning, here is a sentence containing a citation\cite{multifrac}.

Now a little text: \lipsum[1-2]

Here is a paragraph referencing a figure: A phenominological demonstration of this behavior is shown in figure \ref{2010_09_sparse1d}.  Various $D_q$ curves were computed for the logistic map, with different numbers of points used to populate the histogram.  When the histogram is too sparse, the curves asymptotically approach a wide spread of values for $q\to -\infty$.  These values are not related to the actual $D_{-\infty}$ result.

\begin{figure}[!ht]
\includegraphics[width=.9\textwidth]{20100922_progression.png}
\caption[Plots demonstrating the effect of sparse data on $D_q$ convergence.]{Plots of the $D_q$ curve for the logistic map as the total number of points is varied.  There were 10,000 bins in the histogram.  Notice that $D_q$ settles down to the expected curve in the general vacinity of 100 points per cell, although in reality this number is slightly higher, since some cells are empty.  Also notice that although $N_{\{n=1\}}=0$ appears to be a necessary requirement, it is not sufficient.  All these curves were calculated with the naive algorithm. \label{2010_09_sparse1d}}
\end{figure}

\chapter{New Methods}

This is the second chapter, and here is a fact which we support with a citation\cite{coxlittleoshea}.  Many people put the chapter declaration inside the inputted file\cite{Yorke1989}, in order to be able to reorder things very easily\cite{robot_homotopy}.  

\section{A new definition of dimension?}

Will you unite the disparate definitions of the current science? \lipsum[1]

\subsection{A subsection}

% This is still experimental.  I hear that perhaps the full caption should go above the table, in which case we would do:
% \begin{table}[htp]
% \caption[short version of caption]{Longer, more detailed caption, perhaps containing a real description of the table's contents. \label{table:faketable}}
% \begin{tabular}...
\begin{table}[htp]
\caption{Test Table \label{table:faketable}}
\begin{tabular}{|c|c|c|}
\hline
Item 1 & Item 2 & Item 3\\
\hline
1 & 2 & 3 \\
4 & 5 & 6 \\
\hline
\end{tabular}
\begin{minipage}{0.9\textwidth} ~\\ \footnotesize
Here is a table with nothing in it.  This is the caption for the table.  It's only here to make sure that the list of tables is working correctly, and as just a double check on basic table sort of stuff.
\end{minipage}%clunky.  Need to make this look better.
\end{table}

This text here is the ``actual" location of table \ref{table:faketable}.  Since tables are floating objects, the table itself may have moved around so as to better fit on the page.  I used the options \verb-[htp]-, which should attempt to place the table \verb-h-ere, or if that fails, at the \verb-t-op of this or the next page\footnote{I'm not sure if the top option will still do next page now that we are using the single side option for amsbook.}, or if that fails, on a \verb-p-age containing only floats.  The float page gets purged at the end of each section, I think.  You can also use global options to specify that all floats end up in special places, like pages containing only floats.

\subsection{Another Subsection}

Automatically generated nonsense text: \lipsum[3-5]

\begin{sidewayspage}
\begin{table}[h]
\caption{Sideways Table \label{table:sidetable}}
\begin{tabular}{|c|c|c|ccc|}
\hline
Item 1 & Item 2 & Item 3 & more items & more items & more items\\
\hline
\hline
1 & 2 & 3 & x & x & x\\
4 & 5 & 6 & x & x & y\\
\hline
\end{tabular}
\end{table}
\end{sidewayspage}

Here is some text inserted after the sideways table.  Hopefully the table will float, not forcing a break before this text.  If these lines are right at the top of a new page, something might be wrong. \lipsum[1-3]

\backmatter
\bibliographystyle{ieeetr} % I still need to check to see if this is what was fucking with my formatting.
% \bibliography{samplebibfile} %note that this is a separate file
\bibliography{leifbib} %note that this is a separate file

\appendix

% \input{appendix1.tex}
\chapter{Automatically Generated Supplementary Material} %this will be displayed as an appendix, not a chapter.  Appendices are at the same level in the hierarchy as chapters.

\section{Some Sample Material}

Appendix is a strange name.  Did the name for the written material come before the name of the organ?

\lipsum[5-7]

\section{On Why I Should Work Harder}

Let me address that issue with the following classical arguments: \lipsum[1-2]

\chapter{Another Supplement}

\lipsum[1-3]

\end{document}