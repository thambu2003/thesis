\label{section:Parameter estimation}
Parameter estimation \cite{Heinrich04parameterestimation} is the problem of finding the parameters $\theta$ for a set of distributions that best explains the observations $X$. 
The dataset $X$ $=\{x_{i}\}_{i=1}^{|X|}$ can be considered as a set of observations generated independently and identically distributed realizations of a random variable. The parameters $\theta$, depends on the distributions considered, for example, Multinomial distribution, $\theta$ $= \{p_{i}\}_{i=1}^{i=D}$, where D is the cardinality of the possible outcomes.

The joint distribution $P(X,\theta)$ describes the probability of the observations for different combinations of the vector $\theta$. Bayes theorem gives the relationship between the probabilities $X$ and $\theta$ as below:
\begin{equation}
\label{equ:bayes}
P(\theta|X) = \frac{P(X|\theta) P(\theta)}{P(X)} 
\end{equation}
The interpretation of the distributions in Eq. \ref{equ:bayes} is given below:
\begin{equation}
posterior ~\alpha \ liklihood \times prior
\end{equation}
Below, we explain some of the methods for parameter estimation. We will start with simple Maximum Liklihood Estimation (MLE) and then describe how prior belief can be included in the estimation. 

\subsection{MLE}
\label{section:MLE}
MLE is the method of finding the parameters of the model that maximizes the probability of the observations(liklihood) under the resulting distribution. The liklihood is,
\begin{equation}
\ell(\theta|X) = \prod _{i=1}^{i=|X|} P(x_{i}|\theta)
\end{equation}
Because of the product, it is often mathematically convenient to express the liklihood, $\ell$, as the log-liklihood,
\begin{equation}
L(\theta|X) = \sum_{i=1}^{i=|X|} log(P(x_{i}|\theta))
\end{equation}
The MLE can then be formulated as,
\begin{equation}
\label{equ:log-liklihood}
\theta_{ML} = \arg \! \max_\theta(L(\theta|X))
\end{equation}

The parameter $\theta$ then can be estimated by solving the Eq. \ref{equ:log-liklihood} as follows:
\begin{equation}
\frac{\partial}{\partial\theta_{d}} L(\theta|X) = 0; \forall \theta_{d} \in \theta
\end{equation}
As an example, consider a set $X$ of $N$ bernoulli experiments of an unfair coin toss with unknown parameter $\theta$. The probability of the event $c$, for a single experiment, for the random variable(r.v) $C$ is,
\begin{equation}
P(X=x|\theta) = \theta^{x}\cdot\theta^{1-x} 
\end{equation}
where $x=1$ is heads and $x=0$ is tails.
The MLE for $\theta$ can be found by solving Eq. \ref{equ:log-liklihood},
\begin{eqnarray}
L & = & \sum_{i=1}^{i=N}(log(P(X=x|\theta))) \\
 & = & (n^{1}log(P(X=1|\theta))) + n^{0}log(P(X=0|\theta)))
\end{eqnarray}
where $n^{1}$ denotes the number of heads and $n^{0}$, the number of tails.
\begin{equation}
\frac{\partial}{\partial\theta}(L) = \frac{x^{1}}{\theta} - \frac{x^{0}}{1-\theta}=0 
\end{equation}
\begin{equation}
\label{equ:ML}
\theta_{ML} = \frac{x^{1}}{x^{1}+x^{0}} = \frac{x^{1}}{N} 
\end{equation}
which is the ratio of heads to the total number of samples. 

It can be seen from Eq. \ref{equ:ML}, the MLE estimates parameters which best explains the observations. If the observations or the sample dataset(subset of sample space) is not a good representative of the population(sample sapce) then the MLE estimate approach overfits the parameters to the sample dataset.

\subsection{Maximum a posteriori estimation}
Maximum a posteriori (MAP) estimation is similar to MLE but also incorporates a mechanism to add prior belief in the form of a prior distribution. In MAP, the parameters of the model are obtained by maximizing the posterior distribution in Eq. \ref{equ:bayes} with respect to the model parameters $\theta$
\begin{eqnarray}
\theta_{MAP} & = & \arg\!\max_\theta(P(X|\theta) \cdot P(\theta)) \ \ \ \ \mid \scalebox{0.75}{$ P(X) \neq f(\theta)$} \\
& = & \arg\!\max_\theta(\sum_{i=1}^{N}log(P(x_{i}|\theta)) + log(P(\theta)))
\label{equ:MAP}
\end{eqnarray}
Continuing with the coin example as in MLE, the prior distribution $P(\theta)$ is represented by the Beta distribution(explained in Section \ref{section:conjugate priors}) with hyperparameters $\alpha$ and $\beta$ as below:
\begin{eqnarray}
P(\theta) & = & \frac{\theta^{\alpha}\cdot(1-\theta)^{\beta}}{B(\alpha,\beta)} \ \ \ \ \mid \scalebox{0.75}{$B(\alpha,\beta)$ = beta function}
\end{eqnarray}
\begin{equation}
\label{equ:prior}
\frac{\partial}{\partial\theta}P(\theta) = \frac{\alpha-1}{\theta} + \frac{\beta-1}{\theta}
\end{equation}
Substituting Eq. \ref{equ:ML} and \ref{equ:prior} in \ref{equ:MAP} and simplifying, we obtain,
\begin{equation}
\label{equ:MAP estimate}
\theta_{MAP} = \frac{x^{1}+\alpha-1}{N+\alpha-1+\beta-1}
\end{equation}
From the MAP estimate in Eq. \ref{equ:MAP estimate}, we can see that, the addition of prior distribution is just including past experimental results or belief.
The addition of prior belief acts like regularization to the MLE estimate